\documentclass[11pt]{article}

% set 1-inch margins in the document
\usepackage[margin=1in]{geometry}
\usepackage{amsthm}
\usepackage{amsmath}
\usepackage{amssymb,amsfonts}
\theoremstyle{definition}

% include this if you want to import graphics files with /includegraphics
\usepackage{graphicx}
\providecommand{\abs}[1]{\lvert#1\rvert}

% info for header block in upper right hand corner

\newtheorem{problem}{Problem}

\title{HIT --- Cryptography --- Homework 5}

\begin{document}

\maketitle

\begin{problem}
This question concerns the Euler phi function.
\begin{enumerate}
\item Let  $p$ be a prime and $e \ge 1$ an integer. Show that $\phi(p^e) = p^{e-1}(p-1)$.
\item Let $p,q$ be relatively prime. Show that $\phi(pq) = \phi(p)\cdot \phi(q)$. (You may use the Chinese remainder theorem.)
\item Prove Theorem: $N = \prod_ip_i^{e_i}$, $\{p_i\}$ are distinct primes, $\phi(N) = \prod_ip_i^{e_i-1}(p_i-1)$.
\end{enumerate}
\end{problem}

\begin{problem}
Solve the following system of congruences (find $x$ by hand):
\[ 13x \equiv 4 \pmod{99},\;\;\;\;\; 15x \equiv 56 \pmod{101}\]
\end{problem}

\begin{problem}
Compute $[101^{4,800,000,023} \bmod 35]$ (by hand).
\end{problem}

\begin{problem}
Let $N=pq$ be a product of two distinct primes. Show that if $\phi(N)$ and $N$ are known, then it is possible to compute $p$ and $q$ in polynomial time.
\end{problem}


\end{document}